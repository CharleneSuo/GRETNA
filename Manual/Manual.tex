\documentclass[11pt]{article}
%\documentclass[11pt]{book}
\usepackage{amsmath}
\usepackage{amssymb}
\usepackage{makeidx}
\usepackage{graphicx}
\usepackage[colorlinks,linkcolor=red,anchorcolor=pink,citecolor=green]{hyperref}
\usepackage{pifont}
\usepackage{listings}
\usepackage{xcolor}

\usepackage{geometry}
\geometry{left=2em,right=2em}%,top=,bottom=2.5cm}

\usepackage{indentfirst}
\lstset{numbers=none,
		numberstyle=\large,
		keywordstyle=\color{pink},
		commentstyle=\color{green},
		frame=shadowbox;
		escapeinside=``,
		breaklines,
		extendedchars=false,
		xleftmargin=1em,
		xrightmargin=1em,
		aboveskip=1em,
		tabsize=4,
		showspaces=false
		}

\setlength{\parindent}{0em}
\linespread{2}

\title{GRETNA}
\author{Sandy Wang}
\begin{document}
	\maketitle
	\tableofcontents
	\section{Overview}
		\textbf{GRETNA} toolbox has been designed for the \textbf{graph-theoretical network analysis} of fMRI data. 
		It is a suite of MATLAB functions and some MATLAB-based Interface to perform the process of conventional fMRI preprocessing,
		as well as to calculate most frequently used network metrics, 
		like small world, efficiency, degree, betweenness, assortativity, hierarchy, synchronization and modularity.
        \textbf{Please cite}: Wang J, Wang X, Xia M, Liao X, Evans A, He Y (2015)
        GRETNA: a graph theoretical network analysis toolbox for imaging 
        connectomics. Frontiers in human neuroscience, 9.
	\section{Licence}
		\textbf{GRETNA} is distributed under the terms of the GUN General Public Licence as published by the Free Software Foundation (version 3) 
		and the details on "copyleft" can be found at \url{http://www.gnu.org/copyleft/}.
	\section{Prerequisites}
		You need the following to run \textbf{GRETNA} on your computer:
		\begin{itemize}
			\item MATLAB: A high level numerical mathematics environment developed by MathWorks, Inc. Natick, MA, USA.
				GRETNA requires MATLAB2010a or later version
			\item SPM8: SPM is made freely available to the (neuro)imaging community, 
				to promote collaboration and a common analysis scheme across laboratories. 
				The software represents the implementation of the theoretical concepts of Statistical Parametric Mapping in a complete analysis package.
			\item MRICroN's dcm2nii: 
				GRETNA included this package in its distribution. So you do not need download MRICroN's dcm2nii again.
			\item MatlabBGL: MatlabBGL is a MATLAB package for working with graphs. 
				It uses the Boost Graph Library to efficiently implement the graph algorithms.
				GRETNA included this package in its distribution. So you do not need download MatlabBGL again.
			\item PSOM: The pipeline system for GNU Octave and Matlab(r) (PSOM) is a lightweght library to manage complex multi-stage data processing.
				A pipeline is a collection of jobs, i.e. Matlab or Octave codes with a well identified set of options that are using files for inputs and outputs.
				GRETNA included this package in its distribution. So you do not need download PSOM again.
		\end{itemize}
	\section{Installation}
		\textcolor{red}{Warning: Please ensure your GRETNA path do not include blank!.}
		\subsection{Command-line}
			If you do not have write permission for the path of GRETNA, 
			please add GRETNA to MATLAB's path with the following command every time you launch MATLAB:
			\begin{lstlisting}
				>> addpath(genpath("/usr/local/share/software/GRETNA"));
			\end{lstlisting}
			Where \textbf{"/usr/local/share/software/GRETNA"} is the location of your GRETNA.
		\subsection{Interface}
			\begin{figure}
				\begin{center}
					\includegraphics[width=80em]{SetPath.eps}
				\end{center}
			\end{figure}
			To ensure GRETNA is automatically on your MATLAB path in the future, 
			you need launch MATLAB and run the following command to manage your MATLAB path: 
			\begin{lstlisting}
				>> pathtool;
			\end{lstlisting}
			Then:
			\begin{enumerate}
				\item Click "Add with Subfolders..." button and 
					select your GRETNA path, i.e. "/usr/local/software/GRETNA"
				\item Save your change. If you do not have the permission to save your change in GRETNA folder, 
					please save \textbf{pathdef.m} to another location where you will often launch MATLAB.
			\end{enumerate}
    \section{Starting}
			\begin{figure}
				\begin{center}
					\includegraphics[width=40em]{Starting.eps}
				\end{center}
			\end{figure}
			GRETNA include three sub-module:
            \begin{enumerate}
                \item Network Construction: The fMRI pre-processing and 
                    the estimation of ROI-based correlation matrix.
                \item Network Analysis: The estimation of network and 
                    nodal metrics.
                \item Network Comparison: The statistics of network, nodal
                    and edge metrics.
            \end{enumerate}
			you open GRETNA interface by 
			\begin{lstlisting}
				>> gretna;
			\end{lstlisting}

	\section{Toolbar}
		\subsection{Save Default Configure}
			\begin{figure}
				\begin{center}
					\includegraphics[width=50em]{Toolbar_Default.eps}
				\end{center}
			\end{figure}
			You can save your customized configuration for \textbf{GRETNA}, 
			it will be the configure you want every time you open \textbf{GRETNA}.
		\subsection{Manual}
			\begin{figure}
				\begin{center}
					\includegraphics[width=50em]{Toolbar_Manual.eps}
				\end{center}
			\end{figure}
			Click to open this manual.
		\subsection{Load Configure}
			\begin{figure}
				\begin{center}
					\includegraphics[width=50em]{Toolbar_Load.eps}
				\end{center}
			\end{figure}
			To load the configuration you save.
		\subsection{Save Configure}
			\begin{figure}
				\begin{center}
					\includegraphics[width=50em]{Toolbar_Save.eps}
				\end{center}
			\end{figure}
			To save the current configuration.
		\subsection{Run}
			\begin{figure}
				\begin{center}
					\includegraphics[width=50em]{Toolbar_Run.eps}
				\end{center}
			\end{figure}
			To run the pipeline.
		\subsection{Refresh}
			\begin{figure}
				\begin{center}
					\includegraphics[width=50em]{Toolbar_Refresh.eps}
				\end{center}
			\end{figure}
			As usual, the status of pipeline will be refreshed automatically. 
			If you found any error message in MATLAB's command-line window, 
			that is \textbf{not} the errors for pipeline but the errors for the refresh process.
			Please click this button to get the status of pipeline.
		\subsection{Stop}
			\begin{figure}
				\begin{center}
					\includegraphics[width=50em]{Toolbar_Stop.eps}
				\end{center}
			\end{figure}
			Before you close the interfaces of \textbf{GRETNA}, 
			\textbf{please} click this button \textbf{if the pipeline is still running}.
	\section{Input Format}
		For \textbf{GRETNA}, you can import functional image sequences to get network matrices, 
		or you can import network matrices directly. 
		\subsection{Brain Image}
			\textbf{GRETNA} support fMRI sequences with raw DICOM data from scanners or 3D/4D NIfTI-1 format.
			You need select the directory where you store all subjects, 
			and also the \textbf{key word} as what you do in SPM8.

			These fMRI data should be stored in 4 given rule:
			\begin{figure}
				\begin{center}
					\includegraphics[width=80em]{InputImage.eps}
				\end{center}
			\end{figure}
			\begin{enumerate}
				\item The sub-folders of subjects for DICOM.
				\item The sub-folders of subjects for 3D NIfTI-1 files.
				\item The sub-folders of subjects for 4D NIfTI-1 files.
				\item 4D NIfTI-1 files with subjects' name.
			\end{enumerate}		
		\subsection{Network Matrix}
			\textbf{GRETNA} support network matrix with MATLAB's MAT-file or TEXT file.
			For MAT-file, you can use MATLAB's norm or sparse matrix to save your network.
			You can check your data by double-click in GRETNA's interface.

			These files should be stored in 3 given rule:
			\begin{figure}
				\begin{center}
					\includegraphics[width=80em]{NetInput.eps}
				\end{center}
			\end{figure}
			\begin{itemize}
				\item The network matrices with different field name in MAT-file.
				\item The N $\times$ 1 cell of network metrices in MAT-file
				\item The network matrices in TEXT file.
			\end{itemize}		
	\section{Network Construction}
		For \textbf{Slice Timing, Realign, Normalization, Smooth}, you can read SPM manual to know more details.
		\subsection{DICOM to NIfTI}
			\begin{figure}
				\begin{center}
					\includegraphics[width=80em]{DCM2NII.eps}
				\end{center}
			\end{figure}
			\begin{itemize}
				\item \textbf{Time Point}: The number of time points for your data.
			\end{itemize}
		\subsection{Delete Images}
			\begin{figure}
				\begin{center}
					\includegraphics[width=80em]{DeleteImage.eps}
				\end{center}
			\end{figure}
			The first n volumes can be discarded for the signal equilibrium and participants' adaptation to the scanning noise.
			\begin{itemize}
				\item \textbf{The delete type}: Select "Delete" or "Retain" by double-click to choose the type for delete images.
				\begin{itemize}
					\item Select the \textbf{number of image sequences} to delete or retain by double-click.
				\end{itemize}
			\end{itemize}
		\subsection{Slice Timing}
			\begin{figure}
				\begin{center}
					\includegraphics[width=80em]{SliceTiming.eps}
				\end{center}
			\end{figure}
			\begin{itemize}
				\item \textbf{Number of Slices}: The number of slices in one volumn.
				\item \textbf{TR (s)}: The time of repeat of fMRI signal.
				\item \textbf{Slice Order}: The sequence of Slice, e.g. interleaved (bottom $->$ up) [1:2:nslices 2:2:nslices].
				\item \textbf{Reference Slice}: The slice number as reference.
			\end{itemize}
		\subsection{Realign}
			\begin{figure}
				\begin{center}
					\includegraphics[width=50em]{HM.eps}
				\end{center}
			\end{figure}
			You can check subjects' head motion parameter in the "GretnaLogs/HeadMotion" folder.
		\subsection{Normalize EPI}
			\begin{figure}
				\begin{center}
					\includegraphics[width=80em]{NormalizeEPI.eps}
				\end{center}
			\end{figure}
			\begin{itemize}
				\item \textbf{Source Image Path}: The subjects' parent directory of image that is warped to match the templates.
				\item \textbf{Source Image Prefix}: The prefix of image that is warped to match the templates.
			\end{itemize}
			\begin{itemize}
				\item \textbf{Voxel Sizes (mm)}: The voxel size of the written normalised images.
				\item \textbf{Bounding Box}: The bounding box (in mm) of the volume which is to be written. 
			\end{itemize}
		\subsection{Normalize T1}
			\begin{figure}
				\begin{center}
					\includegraphics[width=80em]{Normalize.eps}
				\end{center}
			\end{figure}
			\begin{itemize}
				\item \textbf{T1 Path}: The subjects' parent directory of T1 image.
				\item \textbf{DICOM to NIfTI}: Execute DICOM to NIfTI or not.
				\item \textbf{Coregister}: Execute coregister T1 image to image that is warped to match the templates or not.
				\item \textbf{Segment}: Execute segment T1 image or not.
					\begin{itemize}
						\item \textbf{Source Image Path}: The subjects' parent directory of image that is warped to match the templates.
						\item \textbf{Source Image Prefix}: The prefix of image that is warped to match the templates.
						\item \textbf{T1 Image Prefix}: The prefix of T1 image.
						\item \textbf{Affine Regularisation}: Select "mni" or "estern".
					\end{itemize}
				\item \textbf{Mat Suffix}: The suffix of transation matrix.
			\end{itemize}
			\begin{itemize}
				\item \textbf{Voxel Sizes (mm)}: The voxel size of the written normalised images.
				\item \textbf{Bounding Box}: The bounding box (in mm) of the volume which is to be written. 
			\end{itemize}
		\subsection{Smooth}
			\begin{figure}
				\begin{center}
					\includegraphics[width=80em]{Smooth.eps}
				\end{center}
			\end{figure}
			\begin{itemize}
				\item \textbf{FWHM (mm)}: The full width half maximum of kernel.
			\end{itemize}
		\subsection{Detrend}
			\begin{figure}
				\begin{center}
					\includegraphics[width=80em]{Detrend.eps}
				\end{center}
			\end{figure}
			\begin{itemize}
				\item \textbf{The Degree of Polynomial Curve fitting}: The degree of trend.
				\item \textbf{Remain Mean}: Remain the mean of time courses or not.
			\end{itemize}
		\subsection{Filter}
			\begin{figure}
				\begin{center}
					\includegraphics[width=80em]{Filter.eps}
				\end{center}
			\end{figure}
			\begin{itemize}
				\item \textbf{TR (s)}: The time of repeat of fMRI signal.
				\item \textbf{Band (Hz)}: The frequency band for temporal filtering.
			\end{itemize}
		\subsection{Covariates Regression}
			\begin{figure}
				\begin{center}
					\includegraphics[width=80em]{Regressout.eps}
				\end{center}
			\end{figure}
			\begin{itemize}
				\item \textbf{Global Signal}: Regress out global signal or not.
					\begin{itemize}
						\item \textbf{Brain Mask}: The mask of whole brain.
					\end{itemize}
				\item \textbf{White Matter Signal}: Regress out white matter signal or not.
					\begin{itemize}
						\item \textbf{White Mask}: The mask of white matter.
					\end{itemize}
				\item \textbf{CSF Signal}: Regress out cerebrospinal fluid signal or not.
					\begin{itemize}
						\item \textbf{CSF Mask}: The mask of cerebrospinal fluid.
					\end{itemize}
				\item \textbf{Head Motion}: Regress out head motion parameters or not.
					\begin{itemize}
						\item \textbf{Text Parent Path}: The subjects' parent directory of head motion parameters' text file.
						\item \textbf{Text Prefix}: The prefix of head motion parameters' text file.
						\item \textbf{Add Derivative (12)}: Regress out derivative head motion or not.
					\end{itemize}
			\end{itemize}
		\subsection{Voxel-based Degree}
			\begin{figure}
				\begin{center}
					\includegraphics[width=80em]{VoxelDegree.eps}
				\end{center}
			\end{figure}
			\begin{itemize}
				\item \textbf{Degree Mask}: The mask that include all network nodes you want.
				\item \textbf{Connectional Threshold}: The threshold for correlation coefficient.
				\item \textbf{Connectional Distance}: The euclidean distance between nodes.
			\end{itemize}
            All results of voxel-based degree will be stored in the
            subjects' sub-folder of ``GretnaVoxelDegree'' 
            folder with the following specific label:
			\begin{figure}
				\begin{center}
					\includegraphics[width=80em]{VoxelDegreeResults.eps}
				\end{center}
			\end{figure}
            \begin{itemize}
                \item *wei*, *bin*: using weighted or binary connections to 
                    estimate voxel-based degree.
                \item *pos*, *neg* or *abs*: using positive, negative or
                    absolute connections to estimate voxel-based degree.
                \item *short* or *long*: using short or long connections
                    which were defined in GUI to estimate voxel-based 
                    degree.
                \item *Fisher*: executing Fisher's Z transformation instead
                    of R to estimate voxel-based degree.
                \item *ZScore*: standardizing results with z-score.
            \end{itemize}
		\subsection{Functional Connectivity Matrix}
			\begin{figure}
				\begin{center}
					\includegraphics[width=80em]{FC.eps}
				\end{center}
			\end{figure}
			\begin{itemize}
				\item \textbf{Label Mask}: The template of brain which have different number for different region, e.g. AAL90.
			\end{itemize}
	\section{Network Analysis}
		\subsection{Network Configurations}
			\begin{figure}
				\begin{center}
					\includegraphics[width=80em]{NetworkMetric.eps}
				\end{center}
			\end{figure}
			\begin{itemize}
				\item \textbf{Network Type}: Select "weighted" or "binary" network.
				\item \textbf{Network Member}: Remove the negative value in matrix (Positive), remove the positive value in matrix (Negative) or use the absolute value (Absolute).
				\item \textbf{Threshold Type}: Select the method to cut the network matrices, "sparsity" or "similarity threshold". 
					If DTI's network matrix, select "similarity threshold" and input the threshold of fiber number.
					If BOLD-fMRI's network matrix, select "similarity threshold" for "correlation coefficient" or "r value". 
				\item \textbf{Threshold Range}: Select the range of threshold, it could be one value or a sequence.
				\item \textbf{Random Network}: The number of random network.
			\end{itemize}
		\subsection{Network - Small World}
		\subsection{Network - Efficiency}
		\subsection{Network - Rich Club}
		\subsection{Network - Modularity}
		\subsection{Network - Assortativity}
		\subsection{Network - Hierarchy}
		\subsection{Network - Synchronization}
		\subsection{Node - Degree}
		\subsection{Node - Efficiency}
		\subsection{Node - Betweenness}
	\section{Network Analysis Results}
		All results of network metric will be stored with MAT-file and TEXT-file both. 
		You can pick these metrics from subjects' diretory (e.g. "MAT\_TestMatrix\_VAR\_A") one by one or use the integrated results (Results\_*).
		\begin{figure}
			\begin{center}
				\includegraphics[width=80em]{Results.eps}
			\end{center}
		\end{figure}
		\subsection{Individual Results}
			\subsubsection{Network - Small World}
				\begin{figure}
					\begin{center}
						\includegraphics[width=80em]{SWMat.eps}
					\end{center}
				\end{figure}
				You can load "SWMat.mat" to get the following metrics.
				\begin{itemize}
					\item \textbf{Cp}: Clustering coefficient of network.
						1$\times$N array, N is the number of threshold sequences.
					\item \textbf{Lp}: Shortest path length of network.
						1$\times$N array, N is the number of threshold sequences.
					\item \textbf{nodalCp}: Clustering coefficient of node.
						M$\times$N array, M is the number of nodes, N is the number of threshold sequences.
					\item \textbf{nodalLp}: Shortest path length of node.
						M$\times$N array, M is the number of nodes, N is the number of threshold sequences.
					\item \textbf{Cpzscore}: The z-score of clustering coefficient of network, 
						1$\times$N array, N is the number of threshold sequences.

						The formula is following:
						$$Cpzscore=\frac{Cp-mean(Cprand)}{std(Cprand)}$$
						Cprand is a R$\times$1 array, R is the number of randomized network. 
						It is the clustering coefficient of randomized network.
					\item \textbf{Lpzscore}: The z-score of shortest path length of network,
						1$\times$N array, N is the number of threshold sequences.

						The formula is following:
						$$Lpzscore=\frac{Lp-mean(Lprand)}{std(Lprand)}$$
						Lprand is a R$\times$1 array, R is the number of randomized network. 
						It is the shortest path length of randomized network.
					\item \textbf{Gamma}: Gamma is the ratio of Cp and mean value of Cprand,
						1$\times$N array, N is the number of threshold sequences.
					
						The formula is following:
						$$Gamma=\frac{Cp}{mean(Cprand)}$$
					\item \textbf{Lambda}: Lambda is the ratio of Lp and mean value of Lprand,
						1$\times$N array, N is the number of threshold sequences.
						
						The formula is following:
						$$Lambda=\frac{Lp}{mean(Lprand)}$$
					\item \textbf{Sigma}: Sigma is the ratio of Gamma and Lambda.
						1$\times$N array, N is the number of threshold sequences.
						
						The formula is following:
						$$Sigma=\frac{Gamma}{Lambda}$$
					\item \textbf{aCp}: The AUC (area under curve) of Cp.
					\item \textbf{aLp}: The AUC of Lp.
					\item \textbf{anodalCp}: The AUC of nodalCp.
						M$\times$1 array, M is the number of nodes.
					\item \textbf{anodalLp}: The AUC of nodalLp.
						M$\times$1 array, M is the number of nodes.
					\item \textbf{aCpzscore}: The AUC of Cpzscore.
					\item \textbf{aLpzscore}: The AUC of Lpzscore.
					\item \textbf{aGamma}: The AUC of Gamma.
					\item \textbf{aLambda}: The AUC of Lambda.
					\item \textbf{aSigma}: The AUC of Sigma.
				\end{itemize}
			\subsubsection{Network - Efficiency}
				\begin{figure}
					\begin{center}
						\includegraphics[width=80em]{EFFMat.eps}
					\end{center}
				\end{figure}
				You can load "EFFMat.mat" to get the following metrics.
				\begin{itemize}
					\item \textbf{Eloc}: Local efficiency of network.
						1$\times$N array, N is the number of threshold sequences.
					\item \textbf{Eg}: Global efficiency of network.
						1$\times$N array, N is the number of threshold sequences.
					\item \textbf{nodalEloc}: Local efficiency of node.
						M$\times$N array, M is the number of nodes, N is the number of threshold sequences.
					\item \textbf{nodalEg}: Global efficiency of node.
						M$\times$N array, M is the number of nodes, N is the number of threshold sequences.
					\item \textbf{Eloczscore}: The z-score of local efficiency of network, 
						1$\times$N array, N is the number of threshold sequences.

						The formula is following:
						$$Eloczscore=\frac{Eloc-mean(Elocrand)}{std(Elocrand)}$$
						Cprand is a R$\times$1 array, R is the number of randomized network. 
						It is the local efficiency of randomized network.
					\item \textbf{Egzscore}: The z-score of global efficiency of network,
						1$\times$N array, N is the number of threshold sequences.

						The formula is following:
						$$Egzscore=\frac{Eg-mean(Egrand)}{std(Egrand)}$$
						Egrand is a R$\times$1 array, R is the number of randomized network. 
						It is the global efficiency of randomized network.
					\item \textbf{EGamma}: EGamma is the ratio of Eloc and mean value of Elocrand,
						1$\times$N array, N is the number of threshold sequences.
						
						The formula is following:
						$$EGamma=\frac{Eloc}{mean(Elocrand)}$$
					\item \textbf{ELambda}: ELambda is the ratio of Eg and mean value of Egrand,
						1$\times$N array, N is the number of threshold sequences.
						
						The formula is following:
						$$ELambda=\frac{Eg}{mean(Egrand)}$$
					\item \textbf{ESigma}: ESigma is the ratio of EGamma and ELambda.
						1$\times$N array, N is the number of threshold sequences.
						
						The formula is following:
						$$ESigma=\frac{EGamma}{ELambda}$$
					\item \textbf{aEloc}: The AUC (area under curve) of Eloc.
					\item \textbf{aEg}: The AUC of Eg.
					\item \textbf{anodalEloc}: The AUC of nodalEloc.
						M$\times$1 array, M is the number of nodes.
					\item \textbf{anodalEg}: The AUC of nodalEg.
						M$\times$1 array, M is the number of nodes.
					\item \textbf{aEloczscore}: The AUC of Eloczscore.
					\item \textbf{aEgzscore}: The AUC of Egzscore.
					\item \textbf{aEGamma}: The AUC of EGamma.
					\item \textbf{aELambda}: The AUC of ELambda.
					\item \textbf{aESigma}: The AUC of ESigma.
				\end{itemize}
			\subsubsection{Network - Rich Club}
				\begin{figure}
					\begin{center}
						\includegraphics[width=80em]{RCMat.eps}
					\end{center}
				\end{figure}
				You can load "RCMat.mat" to get the following metrics.
				\begin{itemize}
					\item \textbf{phi\_real}: The rich club coefficient of real network.
						K$\times$N array, K is the number of binary node degree,
                        from 1 to Node-1,
                        N is the number of threshold sequences.
					\item \textbf{phi\_norm}: The normalized rich club coefficient of 
                        real network.
						K$\times$N array, K is the number of binary node degree, 
                        from 1 to Node-1, 
                        N is the number of threshold sequences.
				\end{itemize}
			\subsubsection{Network - Modularity}
				\begin{figure}
					\begin{center}
						\includegraphics[width=80em]{MODMat.eps}
					\end{center}
				\end{figure}
				You can load "MODMat.mat" to get the following metrics.
				\begin{itemize}
					\item \textbf{community\_index}: The community (listed for each node),
						M$\times$N array, M is the number of nodes, N is the number of threshold sequences.
					\item \textbf{number\_of\_module}: The number of module in network,
						1$\times$N array, N is the number of threshold sequences.
					\item \textbf{modularity}: Modularity value of network,
						1$\times$N array, N is the number of threshold sequences.
					\item \textbf{modularity\_zscore}: The z-score of modularity of network,
						1$\times$N array, N is the number of threshold sequences.

						The formula is following:
						$$modularity\_zscore=\frac{modularity-mean(modrand)}{std(modrand)}$$
						modrand is a R$\times$1 array, R is the number of randomized network. 
						It is the modularity of randomized network.
				\end{itemize}
			\subsubsection{Network - Assortativity}
				\begin{figure}
					\begin{center}
						\includegraphics[width=80em]{ASSMat.eps}
					\end{center}
				\end{figure}
				You can load "ASSMat.mat" to get the following metrics.
				\begin{itemize}
					\item \textbf{r}: Assortativity of network,
						1$\times$N array, N is the number of threshold sequences.
					\item \textbf{rzscore}: The z-score of assortativity of network,
						1$\times$N array, N is the number of threshold sequences.

						The formula is following:
						$$rzscore=\frac{r-mean(rrand)}{std(rrand)}$$
						rrand is a R$\times$1 array, R is the number of randomized network. 
						It is the assortativity of randomized network.
				\end{itemize}
			\subsubsection{Network - Hierarchy}
				\begin{figure}
					\begin{center}
						\includegraphics[width=80em]{HIEMat.eps}
					\end{center}
				\end{figure}
				You can load "HIEMat.mat" to get the following metrics.
				\begin{itemize}
					\item \textbf{b}: Hierarchy of network,
						1$\times$N array, N is the number of threshold sequences.
					\item \textbf{bzscore}: The z-score of hierarchy of network,
						1$\times$N array, N is the number of threshold sequences.

						The formula is following:
						$$bzscore=\frac{b-mean(brand)}{std(brand)}$$
						brand is a R$\times$1 array, R is the number of randomized network. 
						It is the hierarchy of randomized network.
				\end{itemize}
			\subsubsection{Network - Synchronization}
				\begin{figure}
					\begin{center}
						\includegraphics[width=80em]{SYNMat.eps}
					\end{center}
				\end{figure}
				You can load "SYNMat.mat" to get the following metrics.
				\begin{itemize}
					\item \textbf{s}: Synchronization of network,
						1$\times$N array, N is the number of threshold sequences.
					\item \textbf{szscore}: The z-score of synchronization of network,
						1$\times$N array, N is the number of threshold sequences.

						The formula is following:
						$$szscore=\frac{s-mean(srand)}{std(srand)}$$
						srand is a R$\times$1 array, R is the number of randomized network. 
						It is the synchronization of randomized network.
				\end{itemize}
			\subsubsection{Node - Degree}
				\begin{figure}
					\begin{center}
						\includegraphics[width=80em]{NodeDMat.eps}
					\end{center}
				\end{figure}
				You can load "NodeDMat.mat" to get the following metrics.
				\begin{itemize}
					\item \textbf{Deg}: The degree of network,
						1$\times$N array, N is the number of threshold sequences.
					\item \textbf{nodalDeg}: The degree number of nodes,
						M$\times$N array, M is the number of nodes, N is the number of threshold sequences.
					\item \textbf{aDeg}: The AUC of Deg,
						1$\times$N array, N is the number of threshold sequences.
					\item \textbf{anodalDeg}: The AUC of nodalDeg,
						M$\times$N array, M is the number of nodes, N is the number of threshold sequences.
				\end{itemize}
			\subsubsection{Node - Efficiency}
				\begin{figure}
					\begin{center}
						\includegraphics[width=80em]{NodeEMat.eps}
					\end{center}
				\end{figure}
				You can load "NodeEMat.mat" to get the following metrics.
				\begin{itemize}
					\item \textbf{Eg}: The global efficiency of network,
						1$\times$N array, N is the number of threshold sequences.
					\item \textbf{nodalEg}: The global efficiency number of nodes,
						M$\times$N array, M is the number of nodes, N is the number of threshold sequences.
					\item \textbf{aEg}: The AUC of Eg,
						1$\times$N array, N is the number of threshold sequences.
					\item \textbf{anodalEg}: The AUC of nodalEg,
						M$\times$N array, M is the number of nodes, N is the number of threshold sequences.
				\end{itemize}
			\subsubsection{Node - Betweenness}
				\begin{figure}
					\begin{center}
						\includegraphics[width=80em]{NodeBMat.eps}
					\end{center}
				\end{figure}
				You can load "NodeBMat.mat" to get the following metrics.
				\begin{itemize}
					\item \textbf{Be}: The betweenness of network,
						1$\times$N array, N is the number of threshold sequences.
					\item \textbf{nodalBe}: The betweenness number of nodes,
						M$\times$N array, M is the number of nodes, N is the number of threshold sequences.
					\item \textbf{aBe}: The AUC of Be,
						1$\times$N array, N is the number of threshold sequences.
					\item \textbf{anodalBe}: The AUC of nodalBe,
						M$\times$N array, M is the number of nodes, N is the number of threshold sequences.
				\end{itemize}
		\subsection{Integrated Results}
			\subsubsection{Network - Small World}
				\begin{figure}
					\begin{center}
						\includegraphics[width=80em]{SmallWorld.eps}
					\end{center}
				\end{figure}
				You can load "SmallWorld.mat" or TEXT file which have the same names of metric to get the following metrics.
				\begin{itemize}
					\item \textbf{Cp(\_All\_Threshold)}: Clustering coefficient of network.
						M$\times$N array, M is the number of subjects, N is the number of threshold sequences.
					\item \textbf{Lp(\_All\_Threshold)}: Shortest path length of network.
						M$\times$N array, M is the number of subjects, N is the number of threshold sequences.
					%\item \textbf{nodalCp\_Node*}: Clustering coefficient number of nodes, "*" is the label number of nodes,
					%	M$\times$N array, M is the number of subjects, N is the number of threshold sequences.
					%\item \textbf{nodalLp\_Node*}: Shortest path length number of nodes, "*" is the label number of nodes,
					%	M$\times$N array, M is the number of subjects, N is the number of threshold sequences.
					\item \textbf{nodalCp\_Thres*}: Clustering coefficient number of nodes, "*" is the label of threshold sequence,
						M$\times$N array, M is the number of subjects, N is the number of nodes.
					\item \textbf{nodalLp\_Thres*}: Shortest path length number of nodes, "*" is the label of threshold sequence,
						M$\times$N array, M is the number of subjects, N is the number of nodes.
					\item \textbf{Cpzscore(\_All\_Threshold)}: The z-score of clustering coefficient of network, 
						M$\times$N array, M is the number of subjects, N is the number of threshold sequences.
						The formula is following:
						$$Cpzscore=\frac{Cp-mean(Cprand)}{std(Cprand)}$$
						Cprand is a R$\times$1 array, R is the number of randomized network. 
						It is the clustering coefficient of randomized network.
					\item \textbf{Lpzscore(\_All\_Threshold)}: The z-score of shortest path length of network,
						M$\times$N array, M is the number of subjects, N is the number of threshold sequences.
						The formula is following:
						$$Lpzscore=\frac{Lp-mean(Lprand)}{std(Lprand)}$$
						Lprand is a R$\times$1 array, R is the number of randomized network. 
						It is the shortest path length of randomized network.
					\item \textbf{Gamma(\_All\_Threshold)}: Gamma is the ratio of Cp and mean value of Cprand,
						M$\times$N array, M is the number of subjects, N is the number of threshold sequences.
						The formula is following:
						$$Gamma=\frac{Cp}{mean(Cprand)}$$
					\item \textbf{Lambda(\_All\_Threshold)}: Lambda is the ratio of Lp and mean value of Lprand,
						M$\times$N array, M is the number of subjects, N is the number of threshold sequences.
						The formula is following:
						$$Lambda=\frac{Lp}{mean(Lprand)}$$
					\item \textbf{Sigma(\_All\_Threshold)}: Sigma is the ratio of Gamma and Lambda.
						M$\times$N array, M is the number of subjects, N is the number of threshold sequences.
						The formula is following:
						$$Sigma=\frac{Gamma}{Lambda}$$
					\item \textbf{aCp}: The AUC (area under curve) of Cp.
						M$\times$1 array, M is the number of subjects.
					\item \textbf{aLp}: The AUC of Lp.
						M$\times$1 array, M is the number of subjects.
					\item \textbf{anodalCp\_All\_Node}: The AUC of nodalCp.
						M$\times$N array, M is the number of subjects, N is the number of nodes.
					\item \textbf{anodalLp\_All\_Node}: The AUC of nodalLp.
						M$\times$N array, M is the number of subjects, N is the number of nodes.
					\item \textbf{aCpzscore}: The AUC of Cpzscore.
						M$\times$1 array, M is the number of subjects.
					\item \textbf{aLpzscore}: The AUC of Lpzscore.
						M$\times$1 array, M is the number of subjects.
					\item \textbf{aGamma}: The AUC of Gamma.
						M$\times$1 array, M is the number of subjects.
					\item \textbf{aLambda}: The AUC of Lambda.
						M$\times$1 array, M is the number of subjects.
					\item \textbf{aSigma}: The AUC of Sigma.
						M$\times$1 array, M is the number of subjects.
				\end{itemize}
			\subsubsection{Network - Efficiency}
				\begin{figure}
					\begin{center}
						\includegraphics[width=80em]{Efficiency.eps}
					\end{center}
				\end{figure}
				You can load "Efficiency.mat" or TEXT file which have the same names of metric to get the following metrics.
				\begin{itemize}
					\item \textbf{Eloc(\_All\_Threshold)}: Local efficiency of network.
						M$\times$N array, M is the number of subjects, N is the number of threshold sequences.
					\item \textbf{Eg(\_All\_Threshold)}: Global efficiency of network.
						M$\times$N array, M is the number of subjects, N is the number of threshold sequences.
					%\item \textbf{nodalEloc\_Node*}: Local efficiency number of nodes, "*" is the label number of nodes,
					%	M$\times$N array, M is the number of subjects, N is the number of threshold sequences.
					%\item \textbf{nodalEg\_Node*}: Global efficiency number of nodes, "*" is the label number of nodes,
					%	M$\times$N array, M is the number of subjects, N is the number of threshold sequences.
					\item \textbf{nodalEloc\_Thres*}: Local efficiency number of nodes, "*" is the label number of threshold sequences,
						M$\times$N array, M is the number of subjects, N is the number of nodes.
					\item \textbf{nodalEg\_Thres*}: Global efficiency number of nodes, "*" is the label number of threshold sequences,
						M$\times$N array, M is the number of subjects, N is the number of nodes.
					\item \textbf{Eloczscore(\_All\_Threshold)}: The z-score of local efficiency of network, 
						M$\times$N array, M is the number of subjects, N is the number of threshold sequences.
						The formula is following:
						$$Eloczscore=\frac{Eloc-mean(Elocrand)}{std(Elocrand)}$$
						Cprand is a R$\times$1 array, R is the number of randomized network. 
						It is the local efficiency of randomized network.
					\item \textbf{Egzscore(\_All\_Threshold)}: The z-score of global efficiency of network,
						M$\times$N array, M is the number of subjects, N is the number of threshold sequences.
						The formula is following:
						$$Egzscore=\frac{Eg-mean(Egrand)}{std(Egrand)}$$
						Egrand is a R$\times$1 array, R is the number of randomized network. 
						It is the global efficiency of randomized network.
					\item \textbf{EGamma(\_All\_Threshold)}: EGamma is the ratio of Eloc and mean value of Elocrand,
						M$\times$N array, M is the number of subjects, N is the number of threshold sequences.
						The formula is following:
						$$EGamma=\frac{Eloc}{mean(Elocrand)}$$
					\item \textbf{ELambda(\_All\_Threshold)}: ELambda is the ratio of Eg and mean value of Egrand,
						M$\times$N array, M is the number of subjects, N is the number of threshold sequences.
						The formula is following:
						$$ELambda=\frac{Eg}{mean(Egrand)}$$
					\item \textbf{ESigma(\_All\_Threshold)}: ESigma is the ratio of EGamma and ELambda.
						M$\times$N array, M is the number of subjects, N is the number of threshold sequences.
						The formula is following:
						$$ESigma=\frac{EGamma}{ELambda}$$
					\item \textbf{aEloc}: The AUC (area under curve) of Eloc.
						M$\times$1 array, M is the number of subjects.
					\item \textbf{aEg}: The AUC of Eg.
						M$\times$1 array, M is the number of subjects.
					\item \textbf{anodalEloc\_All\_Node}: The AUC of nodalEloc.
						M$\times$N array, M is the number of subjects, N is the number of nodes.
					\item \textbf{anodalEg\_All\_Node}: The AUC of nodalEg.
						M$\times$N array, M is the number of subjects, N is the number of nodes.
					\item \textbf{aEloczscore}: The AUC of Eloczscore.
						M$\times$1 array, M is the number of subjects.
					\item \textbf{aEgzscore}: The AUC of Egzscore.
						M$\times$1 array, M is the number of subjects.
					\item \textbf{aEGamma}: The AUC of EGamma.
						M$\times$1 array, M is the number of subjects.
					\item \textbf{aELambda}: The AUC of ELambda.
						M$\times$1 array, M is the number of subjects.
					\item \textbf{aESigma}: The AUC of ESigma.
						M$\times$1 array, M is the number of subjects.
				\end{itemize}
			\subsubsection{Network - Rich Club}
				\begin{figure}
					\begin{center}
						\includegraphics[width=80em]{RichClub.eps}
					\end{center}
				\end{figure}
				You can load "RichClub.mat" or TEXT file which have the same names of 
                metric to get the following metrics.
				\begin{itemize}
					\item \textbf{phi\_real\_Thres*}: The rich club coefficient of 
                        real network,
                        "*" is the label number of threshold sequences, 
                        M$\times$K array, M is the number of subjects, 
                        K is the number of binary degree, from 1 to Node-1.
					\item \textbf{phi\_norm\_Thres*}: The normalized rich club coefficient of 
                        real network,
                        "*" is the label number of threshold sequences, 
                        M$\times$K array, M is the number of subjects, 
                        K is the number of binary degree, from 1 to Node-1.
                \end{itemize}
			\subsubsection{Network - Modularity}
				\begin{figure}
					\begin{center}
						\includegraphics[width=80em]{Modularity.eps}
					\end{center}
				\end{figure}
				You can load "Modularity.mat" or TEXT file which have the same names of metric to get the following metrics.
				\begin{itemize}
					\item \textbf{community\_index\_Thres*}: The community (listed for each node), "*" is the number of threshold sequences,
						M$\times$N array, M is the number of subjects, N is the number of nodes.
					\item \textbf{number\_of\_module}: The number of module in network,
						M$\times$N array, M is the number of subjects, N is the number of threshold sequences.
					\item \textbf{modularity}: Modularity value of network,
						M$\times$N array, M is the number of subjects, N is the number of threshold sequences.
					\item \textbf{modularity\_zscore}: The z-score of modularity of network,
						M$\times$N array, M is the number of subjects, N is the number of threshold sequences.

						The formula is following:
						$$modularity\_zscore=\frac{modularity-mean(modrand)}{std(modrand)}$$
						modrand is a R$\times$1 array, R is the number of randomized network. 
						It is the modularity of randomized network.
				\end{itemize}
			\subsubsection{Network - Assortativity}
				\begin{figure}
					\begin{center}
						\includegraphics[width=80em]{Assortativity.eps}
					\end{center}
				\end{figure}
				You can load "Assortativity.mat" or TEXT file which have the same names of metric to get the following metrics.
				\begin{itemize}
					\item \textbf{r}: Assortativity of network,
						M$\times$N array, M is the number of subjects, N is the number of threshold sequences.
					\item \textbf{rzscore}: The z-score of assortativity of network,
						M$\times$N array, M is the number of subjects, N is the number of threshold sequences.

						The formula is following:
						$$rzscore=\frac{r-mean(rrand)}{std(rrand)}$$
						rrand is a R$\times$1 array, R is the number of randomized network. 
						It is the assortativity of randomized network.
				\end{itemize}
			\subsubsection{Network - Hierarchy}
				\begin{figure}
					\begin{center}
						\includegraphics[width=80em]{Hierarchy.eps}
					\end{center}
				\end{figure}
				You can load "Hierarchy.mat" or TEXT file which have the same names of metric to get the following metrics.
				\begin{itemize}
					\item \textbf{b}: Hierarchy of network,
						M$\times$N array, M is the number of subjects, N is the number of threshold sequences.
					\item \textbf{bzscore}: The z-score of hierarchy of network,
						M$\times$N array, M is the number of subjects, N is the number of threshold sequences.

						The formula is following:
						$$bzscore=\frac{b-mean(brand)}{std(brand)}$$
						brand is a R$\times$1 array, R is the number of randomized network. 
						It is the hierarchy of randomized network.
				\end{itemize}
			\subsubsection{Network - Synchronization}
				\begin{figure}
					\begin{center}
						\includegraphics[width=80em]{Synchronization.eps}
					\end{center}
				\end{figure}
				You can load "Synchronization.mat" or TEXT file which have the same names of metric to get the following metrics.
				\begin{itemize}
					\item \textbf{s}: Synchronization of network,
						M$\times$N array, M is the number of subjects, N is the number of threshold sequences.
					\item \textbf{szscore}: The z-score of synchronization of network,
						M$\times$N array, M is the number of subjects, N is the number of threshold sequences.

						The formula is following:
						$$szscore=\frac{s-mean(srand)}{std(srand)}$$
						srand is a R$\times$1 array, R is the number of randomized network. 
						It is the synchronization of randomized network.
				\end{itemize}
			\subsubsection{Node - Degree}
				\begin{figure}
					\begin{center}
						\includegraphics[width=80em]{NodeDegree.eps}
					\end{center}
				\end{figure}
				You can load "NodeDegree.mat" or TEXT file which have the same names of metric to get the following metrics.
				\begin{itemize}
					\item \textbf{Deg}: The degree of network,
						M$\times$N array, M is the number of subjects, N is the number of threshold sequences.
					%\item \textbf{nodalDeg\_Node*}: The degree number of nodes, "*" is the label number of nodes,
					%	M$\times$N array, M is the number of subjects, N is the number of threshold sequences.
					\item \textbf{nodalDeg\_Thres*}: The degree number of nodes, "*" is the label number of threhold sequences,
						M$\times$N array, M is the number of subjects, N is the number of nodes.
					\item \textbf{aDeg}: The AUC (area under curve) of Deg,
						M$\times$1 array, M is the number of subjects.
					%\item \textbf{anodalDeg\_Node*}: The AUC of nodalDeg, "*" is the label number of nodes,
					%	M$\times$1 array, M is the number of subjects.
					\item \textbf{anodalDeg\_All\_Threshold}: The AUC of nodalDeg,
						M$\times$N array, M is the number of subjects, N is the number of nodes.
				\end{itemize}
			\subsubsection{Node - Efficiency}
				\begin{figure}
					\begin{center}
						\includegraphics[width=80em]{NodeEfficiency.eps}
					\end{center}
				\end{figure}
				You can load "NodeEfficiency.mat" or TEXT file which have the same names of metric to get the following metrics.
				\begin{itemize}
					\item \textbf{Eg}: The global efficiency of network,
						M$\times$N array, M is the number of subjects, N is the number of threshold sequences.
					%\item \textbf{nodalEg\_Node*}: The global efficiency number of nodes, "*" is the label number of nodes,
					%	M$\times$N array, M is the number of subjects, N is the number of threshold sequences.
					\item \textbf{nodalEg\_Thres*}: The global efficiency number of nodes, "*" is the label number of threshold sequences,
						M$\times$N array, M is the number of subjects, N is the number of nodes.
					\item \textbf{aEg}: The AUC (area under curve) of Eg,
						M$\times$1 array, M is the number of subjects.
					%\item \textbf{anodalEg\_Node*}: The AUC of nodalEg, "*" is the label number of nodes,
					%	M$\times$1 array, M is the number of subjects.
					\item \textbf{anodalEg\_All\_Threshold}: The AUC of nodalEg,
						M$\times$N array, M is the number of subjects, N is the number of nodes.
				\end{itemize}
			\subsubsection{Node - Betweenness}
				\begin{figure}
					\begin{center}
						\includegraphics[width=80em]{NodeBetweenness.eps}
					\end{center}
				\end{figure}
				You can load "NodeBetweenness.mat" or TEXT file which have the same names of metric to get the following metrics.
				\begin{itemize}
					\item \textbf{Be}: The betweenness of network,
						M$\times$N array, M is the number of subjects, N is the number of threshold sequences.
					%\item \textbf{nodalBe\_Node*}: The betweenness number of nodes, "*" is the label number of nodes,
					%	M$\times$N array, M is the number of subjects, N is the number of threshold sequences.
					\item \textbf{nodalBe\_Thres*}: The betweenness number of nodes, "*" is the label number of threshold sequences,
						M$\times$N array, M is the number of subjects, N is the number of nodes.
					\item \textbf{aBe}: The AUC of Be,
						M$\times$1 array, M is the number of subjects.
					%\item \textbf{anodalBe\_Node*}: The AUC of nodalBe, "*" is the label number of nodes,
					%	M$\times$1 array, M is the number of subjects.
					\item \textbf{anodalBe\_All\_Threshold}: The AUC of nodalBe,
						M$\times$N array, M is the number of subjects, N is the number of nodes.
				\end{itemize}
    \section{Network Comparison}
        \subsection{Network and Node}
            GRETNA provided a statistical interface based on the integrated
            network and nodal results obtained from ``network analysis''
            \subsubsection{One Sample T-test}
				\begin{figure}
					\begin{center}
						\includegraphics[width=80em]{StatOneSample.eps}
					\end{center}
				\end{figure}
                \begin{itemize}
                    \item Sample I: A text file of network or nodal
                        metrics obtained from network analysis. This file
                        includes a M$\times$N array, M is the number of
                        samples and N is the number of statistical tests.
                        The numbers of samples and statistical tests are
                        shown in Interface with the format [M]\{N\}, 
                        e.g. [16]\{90\}
                    \item Covariate I: A text file of covariate. This file
                        includes a M$\times$C array, M is the number of 
                        samples and C is the number of covariate, e.g. 
                        age, gender and brain size. A sample file 
                        corresponds to a covariate file.
                    \item Base: The base of one sample t-test.
                \end{itemize}
            \subsubsection{Two Sample T-test}
				\begin{figure}
					\begin{center}
						\includegraphics[width=80em]{StatTwoSample.eps}
					\end{center}
				\end{figure}
                \begin{itemize}
                    \item Sample I: A text file of network or nodal
                        metrics obtained from network analysis. This file
                        includes a M1$\times$N1 array, M1 is the number of
                        samples and N1 is the number of statistical tests.
                        The numbers of samples and statistical tests are
                        shown in Interface with the format [M1]\{N1\}, 
                        e.g. [16]\{90\}
                    \item Sample II: A text file of network or nodal
                        metrics obtained from network analysis. This file
                        includes a M2$\times$N2 array, M2 is the number of
                        samples and N2 is the number of statistical tests.
                        The numbers of samples and statistical tests are
                        shown in Interface with the format [M2]\{N2\}, 
                        e.g. [24]\{90\}
                    \item Covariate I: A text file of covariate. This file
                        includes a M1$\times$C array, M1 is the number of 
                        samples and C is the number of covariate, e.g. 
                        age, gender and brain size. A sample file 
                        corresponds to a covariate file.
                    \item Covariate II: A text file of covariate. This file
                        includes a M2$\times$C array, M2 is the number of 
                        samples and C is the number of covariate, e.g. 
                        age, gender and brain size. A sample file 
                        corresponds to a covariate file.
                \end{itemize}
            \subsubsection{Paired T-test}
				\begin{figure}
					\begin{center}
						\includegraphics[width=80em]{StatPaired.eps}
					\end{center}
				\end{figure}
                \begin{itemize}
                    \item Measure I: A text file of network or nodal
                        metrics obtained from network analysis. This file
                        includes a M$\times$N array, M is the number of
                        samples and N is the number of statistical tests.
                        The numbers of samples and statistical tests are
                        shown in Interface with the format [M]\{N\}, 
                        e.g. [16]\{90\}
                    \item Measure II: A text file of network or nodal
                        metrics obtained from network analysis. This file
                        includes a M$\times$N array, M is the number of
                        samples and N is the number of statistical tests.
                        The numbers of samples and statistical tests are
                        shown in Interface with the format [M]\{N\}, 
                        e.g. [16]\{90\}
                    \item Covariate I: A text file of covariate. This file
                        includes a M$\times$C array, M is the number of 
                        samples and C is the number of covariate, e.g. 
                        cognitive scores. A sample file 
                        corresponds to a covariate file.
                    \item Covariate II: A text file of covariate. This file
                        includes a M$\times$C array, M is the number of 
                        samples and C is the number of covariate, e.g. 
                        cognitive scores. A sample file corresponds to a 
                        covariate file.
                \end{itemize}
            \subsubsection{One-way ANCOVA}
				\begin{figure}
					\begin{center}
						\includegraphics[width=80em]{StatANCOVA.eps}
					\end{center}
				\end{figure}
                \begin{itemize}
                    \item Sample X: A text file of network or nodal
                        metrics obtained from network analysis. This file
                        includes a MX$\times$N array, MX is the number of
                        samples and N is the number of statistical tests.
                        The numbers of samples and statistical tests are
                        shown in Interface with the format [MX]\{N\}, 
                        e.g. [16]\{90\}
                    \item Covariate X: A text file of covariate. This file
                        includes a MX$\times$C array, MX is the number of 
                        samples and C is the number of covariate, e.g. 
                        cognitive scores. A sample file 
                        corresponds to a covariate file.
                \end{itemize}
            \subsubsection{One-way ANCOVA (Repeated Measures)}
				\begin{figure}
					\begin{center}
						\includegraphics[width=80em]{StatANCOVA_Repeated.eps}
					\end{center}
				\end{figure}
                \begin{itemize}
                    \item Measure X: A text file of network or nodal
                        metrics obtained from network analysis. This file
                        includes a M$\times$N array, M is the number of
                        samples and N is the number of statistical tests.
                        The numbers of samples and statistical tests are
                        shown in Interface with the format [M]\{N\}, 
                        e.g. [16]\{90\}
                    \item Covariate X: A text file of covariate. This file
                        includes a M$\times$C array, M is the number of 
                        samples and C is the number of covariate, e.g. 
                        cognitive scores. A sample file 
                        corresponds to a covariate file.
                \end{itemize}
            \subsubsection{Correlation Analysis}
				\begin{figure}
					\begin{center}
						\includegraphics[width=80em]{StatCorr.eps}
					\end{center}
				\end{figure}
                \begin{itemize}
                    \item Sample I: A text file of network or nodal
                        metrics obtained from network analysis. This file
                        includes a M$\times$N array, M is the number of
                        samples and N is the number of statistical tests.
                        The numbers of samples and statistical tests are
                        shown in Interface with the format [M]\{N\}, 
                        e.g. [16]\{90\}
                    \item Covariate I: A text file of covariate. This file
                        includes a C$\times$M array, M is the number of 
                        samples and C is the number of covariate, e.g. 
                        gender. A sample file 
                        corresponds to a covariate file.
                    \item Correlation Seed Series: A text file of correlation
                        seed series. This file
                        includes a M$\times$1 array, M is the number of 
                        samples, e.g. age. A sample file 
                        corresponds to a seed file.
                \end{itemize}
            \subsubsection{Output}
				\begin{figure}
					\begin{center}
						\includegraphics[width=50em]{StatOutput.eps}
					\end{center}
				\end{figure}
                \begin{itemize}
                    \item Output Dir: The output folder.
                    \item P: The threshold for p values to obtain the thresholded 
                        statistical values.
                    \item Correct Method: The multiple comparison correction method, 
                        including FDR and Bonferoni.
                    \item Prefix: The prefix of output file for statistical 
                        values and p values.
                \end{itemize}
        \subsection{Edge}
            GRETNA provided a interface for edge analysis.
            \subsubsection{Averaged (Functional)}
				\begin{figure}
					\begin{center}
						\includegraphics[width=80em]{EdgeAveraged.eps}
					\end{center}
				\end{figure}
                \begin{itemize}
                    \item Group I: A group of network matrices with the same input format 
                        during network analysis.
                    \item Threshold Type: ``Similarity threshold'', ``Sparsity'' or LANS.
                    \item Threshold Value: 0.05 (default).
                    \item Network Member: ``Orignal'', ``Positive'', ``Negative'' or
                        ``Absolute'', the network member of output group-averaged network.
                    \item Output Dir: The path for output files.
                    \item Prefix: The prefix of output files. This prodedure would generate
                        two output files, e.g. Edge\_Avg.txt and Edge\_B.txt. They are 
                        group-averaged weighted network and the corresponding binary network.
                \end{itemize}
            \subsubsection{Backbone (Structural)}
				\begin{figure}
					\begin{center}
						\includegraphics[width=80em]{EdgeBackbone.eps}
					\end{center}
				\end{figure}
                \begin{itemize}
                    \item Group I: A group of network matrices with the same input format 
                        during network analysis.
                    \item Threshold: 0.25 (default). The probability of edge across networks.
                    \item Output Dir: The path for output files.
                    \item Prefix: The prefix of output files. This prodedure will generate
                        two output files, e.g. Edge\_Backbone.txt and Edge\_B.txt. They are 
                        backbone network and the corresponding binary network.
                \end{itemize}
            \subsubsection{One-sample T-test}
				\begin{figure}
					\begin{center}
						\includegraphics[width=80em]{EdgeOneSample.eps}
					\end{center}
				\end{figure}
                \begin{itemize}
                    \item Group I: A group of network matrices with the same input format 
                        during network analysis.
                    \item Base: The base value for one-sample t-tests.
                    \item Covariates I: A text file of covariates.
                    \item Correct Method: ``None'', ``FDR'', ``Bonferroni''or ``NBS''.
                    \item Threshold: Edge P value, FDR q or uncorrected P value for NBS.
                    \item Output Dir: The path for output files.
                    \item Prefix: The prefix of output files. This prodedure will generate
                        two output files, e.g. Edge\_P.txt, Edge\_T.txt and Edge\_B.txt. 
                        They are p value network, thresholded t value network and the 
                        corresponding binary network. For NBS correction, an Edge\_Comnet.mat
                        will be created to indicate all components and the corresponding 
                        corrected p value, the threshold of corrected p value is 0.05.
                \end{itemize}
            \subsubsection{Two-sample T-test}
				\begin{figure}
					\begin{center}
						\includegraphics[width=80em]{EdgeTwoSample.eps}
					\end{center}
				\end{figure}
                \begin{itemize}
                    \item Group I: A group of network matrices with the same input format 
                        during network analysis.
                    \item Group II: The other group of network matrices with the same input 
                        format during network analysis.
                    \item Covariates I: A text file of covariates.
                    \item Covariates II: The other text file of covariates.
                    \item Correct Method: ``None'', ``FDR'', ``Bonferroni''or ``NBS''.
                    \item Threshold: Edge P value, FDR q or uncorrected P value for NBS.
                    \item Output Dir: The path for output files.
                    \item Prefix: The prefix of output files. This prodedure will generate
                        two output files, e.g. Edge\_P.txt, Edge\_T.txt and Edge\_B.txt. 
                        They are p value network, thresholded t value network and the 
                        corresponding binary network. For NBS correction, an Edge\_Comnet.mat
                        will be created to indicate all components and the corresponding 
                        corrected p value, the threshold of corrected p value is 0.05.
                \end{itemize}
\end{document}
